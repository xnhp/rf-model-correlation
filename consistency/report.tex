\documentclass[10pt]{article}
% \usepackage{alifeconf}
% \usepackage{a4wide}
\usepackage[hmargin=3cm,vmargin=3cm,nohead]{geometry}

% \usepackage{geometry}

% my packages
% \usepackage{zsfg}
\usepackage{csquotes}
\usepackage[font=footnotesize]{subcaption}
\usepackage[font=footnotesize]{caption}
\usepackage{enumitem}
\usepackage[
% backend=biber,
style=authoryear,
sorting=none
]{biblatex}
% \usepackage{biblatex}
\usepackage{blindtext}
% \usepackage{enumerate}
\addbibresource{MA-Proj.bib}

\renewcommand*{\bibfont}{\small}

% print url if no doi
\renewbibmacro*{doi+eprint+url}{%
  \printfield{doi}%
  \newunit\newblock%
  \iftoggle{bbx:eprint}{%
    \usebibmacro{eprint}%
  }{}%
  \newunit\newblock%
  \iffieldundef{doi}{%
    \usebibmacro{url+urldate}}%
  {}%
}

% \renewcommand{\thesection}{\arabic{section}}
 % \renewcommand\thesection      {\@Roman\c@section}
 % \renewcommand\thesubsection   {\thesection.\@Alph\c@subsection}
 % \renewcommand\thesubsubsection{\thesubsection.\@arabic\c@subsubsection}
 % \renewcommand\theparagraph    {\thesubsubsection.\@alph\c@paragraph}
 % \renewcommand\thesubparagraph {\theparagraph.\@roman\c@subparagraph}

% \renewcommand\thesection{\arabic{section}}
% \renewcommand\thesubsection{\thesection.\arabic{subsection}}

% \def\thesection{\arabic{section}}
\renewcommand{\ttdefault}{cmtt}
\usepackage{titlesec}
\usepackage{titling}
\usepackage{nicefrac}
 % \titleformat{\section}{\LARGE\bfseries}{\thesection}{1em}{}
\usepackage{tcolorbox}
\usepackage[linesnumbered,ruled]{algorithm2e}

\setlength{\algoheightrule}{0pt}
\setlength{\algotitleheightrule}{0pt}
% \SetCommentSty{\familydefault}
\newcommand\mycommfont[1]{\footnotesize \textcolor{teal}{#1}}
\SetCommentSty{mycommfont}

 \setcounter{secnumdepth}{3}% no counters \paragraph and below

 \setlength{\parskip}{1.2ex}
 \setlength{\parindent}{0pt} 

\renewcommand{\baselinestretch}{1.05}


\newcommand{\mat}[1]{
  \mathbf{#1}
}

% bold vectors (instead of arrows)
% \let\oldhat\hat
% \renewcommand{\vec}[1]{\mathbf{#1}}
% \renewcommand{\hat}[1]{\oldhat{\mathbf{#1}}}

\newcommand{\card}[1]{%
  ||#1||
}

\newcommand{\papertitle}[1]{
  \vspace{0.5em}{\Large \enquote{#1}} \vspace{0.5em}
}

\graphicspath{{/} {../figures/}} % Specifies the directory where pictures are stored

% \usepackage[svgnames]{xcolor}

\title{Consistency of Random Forests \\ \vspace{0.5em} \large Project Report}
\pretitle{\begin{center}\Large}
\posttitle{\par\end{center}\vskip 0.5em}
\preauthor{\begin{center}}
\postauthor{\end{center}}
\predate{\begin{center}}
\date{}
\postdate{\end{center}}
\author{Benjamin Moser}

\usepackage{hyperref}
\usepackage{mathpazo}
\usepackage{csvsimple}
% \usepackage{multirow}

\usepackage{amsmath}
\usepackage{amsthm}


\newcommand{\cupdot}{\mathbin{\mathaccent\cdot\cup}}
\theoremstyle{definition}
% \newtheorem{theorem}{Theorem}[section]
\newtheorem{mythm}{Theorem}[section]
\newtheorem{mydef}{Definition}[section]

\usepackage{thmtools}
\declaretheoremstyle[notefont=\bfseries,notebraces={}{},%
headpunct={},postheadspace=1em]{mystyle}
\declaretheorem[style=mystyle,numbered=no,name=Theorem]{mainthm}
\declaretheorem[style=mystyle,numbered=no,name=Hypothesis]{mainhyp}
\declaretheorem[style=mystyle,numbered=no,name=Proposition]{myprop}
\declaretheorem[style=mystyle,numbered=no,name=Lemma]{lemma}

% \newtheorem{def}{Definition}[section]

% , r t   for table of contents
% , r c   and then search term for inserting a citation
\begin{document}
% \raggedbottom


  \maketitle 
  \begin{abstract}

    Random Forest methods are some of the most widely used methods for
    regression and classification learning tasks, and their effectiveness has
    been proven empirically. However, the exact forces determining if, when, and
    why a Random Forest performs well are not yet completely clear.
%
    From a
    statistical point of view, a Random Forest is an estimator of some assumed
    true regression or classification function. A basic notion in statistics is
    that, as we observe more and more data samples, an estimate converges indeed
    to the estimated quantity. This property is called \textit{consistency}.
    %
    In this work, we review proofs of the consistency of Random Forests under
    different conditions as published in \cite{Scornet2015}.
    
    % In this report, we review the main contributions of \cite{Scornet2015}. We
    % begin by recapitulating some basic definitions and introducing the regression
    % learning task. We then proceed to define the Random Forest estimator and
    % what it means for an estimator to be consistent. We review two theorems
    % showing the consistency of Random Forests under different conditions.
    % % 
    % One proof is treated in full and repeated in a narrative manner, with a
    % focus on providing intuition and motivating the imposed assumptions.
    % %
    % For the other theorem we illustrate the overall proof structure and
    % point out the basic ingredients.

    
    % gather background/foundations
    % provide overview of contributions
    % overview and motivation for assumptions made
    % illustrate main ingredients of proofs and interesting points
    % more detailed walkthrough of one proof in a more narrative style with a
    % focus on understanding the overall thing.
  \end{abstract}


  \section{Overview}

  
  In this report, we review the main contributions of \cite{Scornet2015}. We
  begin by recapitulating some basic definitions and introducing the regression
  learning task. We then proceed to define the Random Forest estimator and
  what it means for an estimator to be consistent. We review two theorems
  showing the consistency of Random Forests under different conditions.
  % 
  One proof is treated in full and repeated in a narrative manner, with a
  focus on providing intuition and motivating the imposed assumptions.
  % 
  For the other theorem we illustrate the overall proof structure and
  point out the basic ingredients.

% Random Forest methods are some of the most midely used methods for regression and
% classification learning tasks.
% %
% Intuitively, a Random Forest is an ensemble of
% randomised regression or decision trees. Individual trees are constructed based
% on a random subsample of the training dataset. A tree partitions the data space
% such that the training data points falling in each cell are pure with respect to
% the response variable. The estimate of a tree is the mean (or majority vote) of
% the cell of the query point and the individual tree estimates are aggregated to
% form the forest estimate.

% The effectiveness of Random Forests has been proven empirically. However, the
% exact forces determining if, when and why Random Forests perform well are not
% yet completely clear.
% %
% From a statistical point of view, a Random Forest is an estimator of some
% assumed true regression or classification function. Since the distribution of
% the data is not known in practice, the true function cannot be derived
% explicitly. A basic notion in statistics is that, as we observe more and more
% samples, an estimate converges indeed to the estimated quantity. This property
% is called \textit{consistency}.
% %
% In this work, we seek to prove the consistency of Random Forest estimators. For
% the time being, we consider the Random Forest variant for regression as
% originally introduced in \cite{breiman_RandomForests_2001}.


\section{Preliminaries}


\subsection{Probability \& Statistics}


Since we want to reason about an estimator independent of concrete data,
we make statistical statements based on the in practice unknown
probabilistic distribution of data. In this section, we fixate some basic
language and notation to do that. The definitions are based on
\cite{wasserman_AllStatisticsConcise_2010} and
\cite{gyorfi_DistributionFreeTheoryNonparametric_2002}.

\begin{mydef}{Probability Space}
  A \emph{probability space} is a triple \((\Omega, \Sigma, \mathbb{P})\)
  where
  \begin{itemize}
  \item \(\Omega\) is an arbitrary set modelling the \emph{sample space}
    i.e.~the set of all possible outcomes.
  \item \(\Sigma\) is a \(\sigma\)-algebra
    over \(\Omega\), modelling the set of \emph{events}.
    % \footnote{A
    %   \(\sigma\)-algebra over \(\Omega\) can be thought of a set of subsets of
    %   \(\Omega\), containing \(\Sigma\), such that it is closed under complement
    %   and countable union and intersection.}
    \item \(\mathbb{P}\) is a function
    \(\Sigma \to [0,1]\) such that \(\mathbb{P}(\Omega)=1\) and
    \(\mathbb{P}\left( \bigcup_{i=1}^\infty A_{i} \right) = \sum_{i=1}^\infty
    \mathbb{P}(A_{i})\) for a countable collection of (pairwise disjoint) sets
    in \(\Sigma\), and models the \emph{probability measure}.
  \end{itemize}
\end{mydef}

In the following we assume an underyling probability space implicitly.


% \begin{mydef}[Conditional Probability]
%   Let \(A, B\) be events. Given that
%   \(\mathbb{P}(B) > 0\), the \emph{conditional probability of \(A\) given \(B\)}
%   is \[ \mathbb{P}(A|B) := \frac{\mathbb{P}(A \cap B)}{\mathbb{P}(B)}
%   \]
% \end{mydef}


\begin{mydef}[Random Variable]
  A \textit{random variable} is a quantity
  that depends on a random event, i.e. a function \(\Omega
  \to M\) (commonly, we have \(M=\mathbb{R}\)).
  % that assigns a value to each outcome.
\end{mydef}

Random variables are commonly denoted with uppercase letters. Random variables
extend naturally to random vectors. Any function of a random variable is in turn
a random variable. 

\begin{mydef}[Convergence of Random Variables]
  A sequence of random variables \(X_{n} := \{X_{1}, \dots, X_{n}\}\) is said to
  \emph{converge almost surely} (a.s.) towards random variable \(X\) iff \[
    \mathbb{P}\left(\omega \in \Omega: \lim _{n \rightarrow \infty}
      X_n(\omega)=X(\omega)\right)=1
  \]
\end{mydef}

% In the following, the character of a random variable as a function of an outcome
% is left implicit and we write \[ \mathbb{P}\left(\lim _{n \rightarrow \infty}
%     X_n=X\right)=1
% \]

Since we want to make statements independent of concrete outcomes but rather
about the general behaviour of random variables, we consider their expected
value. For random variables with continuous range, this is based the probability
density function.


\begin{mydef}[Probability Density Function]
  The \textit{probability density function} $f_X$ of a random variable $X$ is a
  nonnegative function such that
  $$
  \mathbb{P}(a < X < b) = \int_a^b f_X(x) dx
  $$
\end{mydef} 

\begin{mydef}[Expected Value]
  The \emph{expected value} (expectation) of random variable \(X\) is
  defined as
  % \[ \mathbb{E}[X] := \int_{-\infty}^\infty x \cdot
  %   \mathbb{P}(X=x) \, dx
  % \]
  
  \[ \mathbb{E}[X] := \int x \cdot
    f_X(x) \, dx
  \]
  where the integral is over the support of $X$. Note that a function \(g(X)\)
  of a random variable \(X\) is a random variable itself and
  $\mathbb{E}[g(X)] = {\int g(x)f_X(x) \, dx}$. To emphasize the random
  variable the function is dependent on, we sometimes mention it in the
  subscript, e.g. $\mathbb{E}_X[g(X)]$.
\end{mydef}

Analogous to probabilities, probability densities and expectations can be
conditioned on the outcome of another random variable.
%
% \begin{mydef}[Conditional Expectation]
%   Let $f$ be the conditional probability density function.
%   The expected value of \(X\) given that \(Y\) takes on value \(y\) is \[
%     \mathbb{E}[X | Y=y] := \int x f_{X|Y}(x|y) \, dx
%   \]
% \end{mydef}
Note that $\mathbb{E}[X|Y = y]$  depends on a concrete observation to
determine a value for \(Y\), thus it is a random variable itself. We
denote it the conditional expecation by \(\mathbb{E}[X|Y]\) as a function of $Y$.
% Similar notation is used for other
% notions below. \^{}dd189d

Let's collect some facts about expectations that will be useful to
us.

\begin{lemma}
  For random variables \(X\) and \(Y\):
  \begin{itemize}
  \item The expectation of the indicator function of an event and consequently
    of a random variable taking on some value is just the probability of that
    happening: \(\mathbb{E}[\mathbb{1}_{X=x}] = \mathbb{P}(X=x)\)
  \item We can relate a bounded random variable to its cumulative distribution
    function. For a random variable $X$ with \(X(\omega) < u ~ \forall \omega \in
    \Omega\) and \(\xi \in \mathbb{R}\), we have \(\mathbb{E}[X] \leq \xi + u
    \mathbb{P}(X > \xi)\)
  \item The rule of iterated expectations states that, for any function \(r\):
    \(\mathbb{E}[r(X,Y)] = \mathbb{E}\left[\mathbb{E}[r(X,Y) | X]\right]\) and,
    in particular, \(\mathbb{E}[X] = \mathbb{E}[\mathbb{E}[X|Y]]\)
  \item Jensen's Inequality: Let $X$ be a random variable such that
    $\mathbb{P}(a \leq X \leq b)=1$. If $g: \mathbb{R} \to
    \mathbb{R}$ is convex on $[a, b]$, then
    $
    g(\mathbb{E}\left[ X \right]) \leq \mathbb{E}\left[ g(X) \right]
    $.
  \item If \(X\) and \(Y\) are independent, then \(\mathbb{E}\left[XY\right] =
    \mathbb{E}\left[X\right] \mathbb{E}\left[Y\right]\)
  \end{itemize}
\end{lemma}

The following quantities consider interactions between random variables.

\begin{mydef}[Variance, Covariance, Correlation] ~
  \begin{itemize}
  \item \emph{Variance}: \(\text{Var}({X})
    := \mathbb{E}[X - \mathbb{E}[X]]^2
    = \mathbb{E}[X^2] - \mathbb{E}[X]^2\)
  \item \emph{Covariance}: \(\text{Cov}(X, Y)
    := \mathbb{E}[(X- \mathbb{E}[X]) (Y- \mathbb{E}[Y]) ]
    = \mathbb{E}[XY] - \mathbb{E}[X] \mathbb{E}[Y]\)
  \item \emph{Correlation}: $\text{Corr}(X, Y) :=
    \frac{\text{Cov}(X,Y)}{(\text{Var}(X)\text{Var}(Y))^{\nicefrac{1}{2}}}$
  \end{itemize}
\end{mydef}
The definitions extend naturally to the conditional case.


\subsection{Learning}

% We formalise the learning task and introduce some error decompositions
% that will be relevant later on.
% The following definitions are based on
% {[}{[}zotero-mdnotes-export/gyorfi\_DistributionFreeTheoryNonparametric\_2002\textbar{}gyorfi{]}{]}
% and
% {[}{[}scornet\_ConsistencyRandomForests\_2015\textbar{}Scornet2015{]}{]}.


\paragraph{Regression} We consider the \emph{regression} task, in which, given some
input vector (\textit{point}) \(x \in [0,1]^p\), we want to predict a response
\(y \in \mathbb{R}\).
%
Generally, we consider input and response to be random variables
\(X\) and \(Y\).
%
We want to find a function 
\(f\) such that \(f(X)\) is an optimal approximation of \(Y\). Note that
\(|f(X)-Y|\) is a random variable, thus we are interested in its
expectation. To assess the quality of \(f\), we use the
\emph{\(L_{2}\)-risk} of \(f\) given by
\(\mathbb{E}\left[|f(X)-Y|^2\right]\), where the square is motivated by
mathematical and computational convenience.
%
In fact, we can derive a
\(L_{2}\)-optimal function explicitly.
%
Let $m(x) := \mathbb{E}\left[ Y | X = x \right]$ be the \emph{regression
  function}. We often also refer to it as a random variable over the
distribution of $X$, that is $m(X) := \mathbb{E}\left[ Y | X \right]$.
%
It can be shown
that, for some candidate estimator $f$ (see \cite{gyorfi_DistributionFreeTheoryNonparametric_2002} sec. 1.1):
\begin{align}
  \mathbb{E}_{(X,Y)}\left[|f(X)-Y|^2\right]
  = \mathbb{E}_X\left[|f(X)-m(X)|^2\right]
  + \mathbb{E}_{(X,Y)}\left[|m(X) - Y|^2\right]
  \label{eq:basic-error-decomp}
\end{align}
% where the expectation is over $X$ and $Y$.
%
% The first term on the right-hand-side is the \emph{\(L_{2}\)-error}
% of \(f\).
One can see that \(f\) is optimal if \(f(x)=m(x)\).

In practice, we do not know the distributions of \(X\) and \(Y\) and hence
cannot use $m$ directly. However, we do have access to samples of their
distributions, commonly also referred to as the \textit{dataset} \(D_{n} := \{
(X_{1}, Y_{1}), \dots, (X_{n}, Y_{n}) \}\). We will consider $D_n$ to be a
random variable.
%
Based on \(D_{n}\), we
may construct an \emph{estimator of $m$}, which we will refer to as
$m_n$. Note that $m_n(X)$ is a function of random variables $X$ and $D_n$ and in
itself a random variable. We assume the samples in $D_n$ to be independently and
identically distributed and $X$ and $Y$ to be \textit{i.i.d.} to any $(X_i,
Y_i)$ in $D_n$.

Similarly to (\ref{eq:basic-error-decomp}), the \textit{\(L_2\)-risk of the
  estimate}  can be decomposed as
\begin{align}
  \label{eq:basic-error-decomp2}
  \mathbb{E}_{D_n} \left[
    \mathbb{E}_{(X,Y)}\left[|m_{n}(X) - Y|^2 \right]
  \right] 
  % &= \mathbb{E}_{(X,Y)}\left[|m_{n}(X) - Y|^2 ~|~ D_{n}  \right] \nonumber
  % \\
  &= \mathbb{E}_{X, D_n}\left[|m_{n}(X) - m(X)|^2\right]
  + \mathbb{E}_{(X,Y)}\left[|m(X)-Y|^2\right]
\end{align}
% The term on the left-hand-side is also referred to as the \textit{generalization
%   error}.
Therefore,
we are interested in 
% we want to find an estimator \(m_{n}\) such that
% its \textit{\(L_{2}\)-error} given by
\(\mathbb{E}\left[|m_{n}(X)- m(X)|^2\right]\)
% is small.
as a central quality measure that is independent of the concrete realisation of
$D_n$ (but note that it is not independent of $n$). This notion of error will be the basis for our definition
the consistency of an estimator later.

Note that we can not directly compute such risks since they
depend on the distributions of \(X\) and \(Y\). As an approximation we
consider the \emph{empirical \(L_{2}\)-risk} of some function \(f\) defined as
\[
  \frac{1}{n} \sum_{i=1}^n |f(X_{i}) - Y_{i}|^2
  \approx 
\mathbb{E}_{(X,Y)}\left[|f(X) - Y|^2\right] 
\]
Minimising over the set of all functions
will likely fit \(D_{n}\) perfectly and result in an estimator that is unlikely to
generalise to the entire distribution of \((X, Y)\). Due to this, we
restrict ourselves to a \emph{hypothesis class} of functions
\(\mathcal{F}_{n}\) and consider \[
m_{n} 
:= 
\arg\min_{f \in \mathcal{F}_{n}}
\frac{1}{n} \sum_{i=1}^n |f(X_{i}) - Y_{i}|^2
\] Indeed, any learning method --- such as Random Forests ---
characterise a hypothesis class that they are algorithmically able to
cover. For Random Forests in particular, \(\mathcal{F}_{n}\) will depend
on \(n\) as well as the randomness that goes into constructing
individual trees.

\paragraph{Estimation and approximation error}
The error of an estimator can be decomposed into different aspects. First, we
may not have found the best possible choice in \(\mathcal{F}_{n}\). Second,
\(\mathcal{F}_{n}\) itself may be too restrictive to allow a close estimate to
the regression function \(m\).
Starting from eq. (\ref{eq:basic-error-decomp2}), one can write
%
\begin{align*}
\mathbb{E}\left[|m_{n}(X) - m(X)|^2\right] &= 
\mathbb{E}\left[|m_{n}(X) - Y|^2 ~|~ D_{n}  \right] 
-
\mathbb{E}\left[|m(X)-Y|^2\right] \\
&=  
\left(
\mathbb{E}\left[|m_{n}(X) - Y|^2 ~|~ D_{n}  \right]
- 
\inf_{f \in \mathcal{F}_{n}} \mathbb{E}\left[| f(X) -Y |^2\right]  \
\right)
 \\ 
& \hspace{1cm}+ \left(
\inf_{f \in \mathcal{F}_{n}} \mathbb{E}\left[| f(X) -Y |^2\right]  -
\mathbb{E}\left[|m(x)-Y|^2\right]\right)
\end{align*}
%
The first term is referred to as the \emph{estimation error} and
quantifies, in terms of error, the distance from the given estimator to
the best possible estimator. The second term is the \emph{approximation
error} and quantifies the distance of the best possible estimator to the
(true) regression function. We will be seeing very similar notions later on, and
often such decompositions will be the first step towards showing consistency.

\paragraph{Bias-Variance-Covariance decomposition}
% TODO cite ueda
Alternatively, the risk can be divided into components of bias and
variance (see \cite{ueda_GeneralizationErrorEnsemble_1996}). The \textit{bias} measures how well the average estimate of the
learning algorithm approximates the true function. The \textit{variance}
measures the variation of the estimate across different training sets of the
same size.
%
Assume there is a functional relationship between $X_i$ and $Y_i$ in the form of
$Y_i = g(X_i) + \varepsilon$ where $\varepsilon$ is noise with
$\mathbb{E}[\varepsilon]=0$ and $\text{Var}(\varepsilon) = \sigma^2 <
\infty$.
Then
\begin{align*}
  \mathbb{E}\left[ |m_{n}(X) - Y |^2 \right]  &=\mathbb{E}_{X}\left[
  \text{Var}(m_{n}|X) + \text{Bias}(m_{n}|X)^2 
                                                \right] + \sigma^2 \\
  \text{where} \hspace{1cm} &
                              \text{Var}(m_{n}|X) :=  \mathbb{E}_{D_{n}}\left[  (m_{n}(X) -  \mathbb{E}_{D_{n}}\left[ m_{n}(X) \right])^2  \right] \\
  & \text{Bias}(m_{n}|X) :=  \mathbb{E}_{D_{n}} \left[ m_{n}(X) \right] - g(X)
\end{align*}

Recall that Random Forests are ensembles of trees. When is an ensemble better
than a single tree? Certainly, if the individual tree estimators are very
similar, we can not expect to gain much over using just a single tree. The individual
trees ought to be \textit{diverse}.
%
For ensemble methods that combine individual
estimators $m_{n}^{(i)}$ for $i \in [M]$ into an ensemble estimator $m_n$ by
means of $m_{n}(X) = \sum_{i=1}^M w_{i} m_{n}^{(i)}(X)$ where the weights $w_{i}$ are nonnegative
and sum to one, the
variance term can be decomposed further into the variance of individual
estimators and the pairwise covariance between estimators.
%
\begin{align*}
  \mathbb{E}\left[ |m_{n}(X) - Y |^2\right]   &=  \mathbb{E}_{X}\left[ 
                                                \frac{1}{M} \overline{\text{Var}}(X) 
                                                + \left( 1-\frac{1}{M} \right) \overline{\text{Cov}}(X) 
                                                + \overline{\text{Bias}}(X)^2
                                                \right]  + \sigma^2
  \\ \text{where} \hspace{1cm} & \overline{\text{Var}}(X) := \frac{1}{M} \sum_{i=1}^M \text{Var}(m_{n}^{(i)}|X) \\ 
                                              & \overline{\text{Cov}}(X) := \frac{1}{M(M-1)} \sum_{i \not= j} \text{Cov}(m_{n}^{(i)}, m_{n}^{(j)}|X) \\
  & 
    \text{Cov}(m_{n}^{(i)}, m_{n}^{(i)}|X) :=  \mathbb{E}_{D_{n}}\left[  \left(m_{n}^{(i)}(X) -  \mathbb{E}_{D_{n}}\left[ m_{n}^{(i)}(X) \right]\right)  \left(m_{n}^{(j)}(X) -  \mathbb{E}_{D_{n}}\left[ m_{n}^{(j)}(X) \right]\right) \right] \\
                                              & \overline{\text{Bias}}(X) := \frac{1}{M} \sum_{i=1}^M \text{Bias}(m_{n}^{(i)}|X)
\end{align*}


% \begin{align*}
%   \mathbb{E}\left[ |m_{n}(X) - m(X)|^2 \right]  = & ~  \overline{\text{bias}}(m_{n})^2 + \frac{1}{M}~ \overline{\text{variance}}(m_{n}) + \left( 1- \frac{1}{M}\right) \overline{\text{covariance}}(m_{n})
% \end{align*}
% where
% \begin{align*}
%   ~\overline{\text{bias}}(m_{n}) &= \nicefrac{1}{M} \sum_{i=1}^M  \mathbb{E}\left[ m_{n}^{(i)}(X) - m(X) \right]  \\
%  ~ \overline{\text{variance}}(m_{n}) &= \frac{1}{M}\sum_{i=1}^M  \mathbb{E}\left[ m_{n}^{(i)}(X) -  \mathbb{E}\left[ m_{n}^{(i)}(X) \right]  \right]^2  \\
%   ~ \overline{\text{covariance}}(m_{n}) &= \frac{1}{M(M-1)} \sum_{i \not= j}^M 
%                               \mathbb{E}\left[ m_{n}^{(i)}(X) -  \mathbb{E}\left[ m_{n}^{(i)}(X) \right]  \right] 
%                               ~ ~ \mathbb{E}\left[ m_{n}^{(j)}(X) -  \mathbb{E}\left[ m_{n}^{(j)}(X) \right]  \right] 
% \end{align*}

This tells us that while combining multiple estimators may mitigate individual
variances, a good ensemble must be also be one in which the individual
estimators are not too similar.
In particular, their errors should exhibit small (absolute) covariance.
Random Forests establish diversity implicitly via the randomness that goes into
constructing individual trees, namely the subsampling prior to the tree
construction and the random selection of candidate split dimensions.



\subsection{Random Forests}

A Random Forest is an ensemble of $M$ randomized regression trees. The
randomness that goes into constructing each tree is captured in identically and
independently distributed random variables $\Theta_1, ..., \Theta_M$. The $k$-th
tree is constructed as follows.

\begin{enumerate}
\item \textit{Initialisation:} Sample $a_{n}$ points from the dataset $D_{n}$ at
  random based on $\Theta_k$ without replacement. These samples define the root
  node of the tree.
\item \textit{Recursion:} Split a node $A$, dividing the points in $A$ into two
  child nodes. A \textit{cut} is a pair $(j, z)$ where $j \in [p]$ indexes the
  dimension in which to perform the split and $z$ the threshold at which to
  split. We will split at $(j^\star,z^\star) = \arg\max_{j \in
    \mathcal{M}_{\text{try}}, (j, z) \in C_{A}} \text{L}(j, z)$ where $L$ is the
  CART split criterion (see def. \ref{def:cart-crit}).

\item \textit{Termination:} Stop as soon as the total number of leaf nodes reaches some given $t_{n}$ (i.e. in
  case of $t_{n} = a_{n}$, each leaf node contains exactly one point).
\end{enumerate}

\begin{mydef}[CART split criterion]
  Let $C_{A}$ be the set of possible cuts in $A$. Let $\mathcal{M}_{try}$ be a
  set of possible dimensions, chosen at random based on $\Theta_j$. The
  \textit{CART Criterion} measures the difference in the (renormalised variance)
  before and after a given split.
  \begin{align*}
    L_{n}(j, z) := & ~ \text{Var}_{\text{before}} - \text{Var}_{\text{after}} \\
    \text{for~ ~} &
                    \text{Var}_{\text{before}} = \frac{1}{N_n(A)} \sum_{i=1}^n\left(Y_i-\bar{Y}_A\right)^2 \mathbb{1}_{\mathbf{X}_i \in A}
                    % = \frac{1}{N_{n}(A)} \sum_{i \in A} (Y_{i} - \bar{Y}_{A})^2
    \\
                  & \text{Var}_{\text{after}} = \frac{1}{N_n(A)} \sum_{i=1}^n\left(Y_i-\bar{Y}_{A_L} \mathbb{1}_{\mathbf{x}_i^{(j)}<z}-\bar{Y}_{A_R} \mathbb{1}_{\mathbf{x}_i^{(j)} \geq z}\right)^2 \mathbb{1}_{ \mathbf{X}_i \in A}
                    % = \frac{1}{N(A)} \sum_{i \in A} (Y_{i} - \tilde Y_{i})^2
  \end{align*}
  \label{def:cart-crit}
\end{mydef}

We refer to \cite{biau_RandomForestGuided_2016} for a comprehensive formulation
of the algorithm. Each level of the tree induced by $\Theta_j$ corresponds
exactly to a partition of the entire data space. Each leaf of the tree
corresponds to a cell in the partition. Let $A_n(x, \Theta_j)$ denote the cell
of $x$ in the tree generated by $\Theta_j$. We say that two points
are \textit{connected} in a tree if they are in the same cell.

\begin{mydef}[Tree and Forest estimates] The individual tree estimates are
  aggregated to form the final Random Forest estimate.
  
  \begin{itemize}
  \item Given a query point $x$, the \textit{estimate of the $j$-th tree},
    denoted by $m_{n}(x, \Theta_{j}, D_{n})$ is the mean response of the
    cell of $x$.

    \item The \textit{finite forest estimate} is the average over all individual tree
    estimates, that is $$m_{M, n}(x, \Theta_{1..M}, D_{n}) := \frac{1}{M}
    \sum_{j=1}^M m_{n}(x, \Theta_{j}, D_{n})$$

  \item In practice, $M$ can be chosen arbitrarily large and indeed the finite
    forest estimate converges almost surely to its expectation as the number of
    trees $M$ grows (see \cite{biau_RandomForestGuided_2016}). Due to this, in
    the following, we will consider the \textit{infinite forest estimate}
    $m_{n}(x) = \mathbb{E}_\Theta \left[m_{n}(x, \Theta, D_{n})\right]$.
  \end{itemize}
\end{mydef}


\subsection{Consistency}

As established earlier, a good estimate is one with small $L_{2}$-error.
In many methods, the quality of the estimate $m_{n}$ depends on the
dataset, in particular its size $n$. As such, we are interested in the question
whether, as $n$ grows, the estimate converges indeed to the optimal 
regression function $m$. This is equivalent to saying that the $L_{2}$-error of the
estimate vanishes as $n$ grows to infinity. This property is called
\textit{consistency}.

\begin{mydef}[Consistency]
  A sequence of regression function estimates $\{ m_{n} \}_{n}$ is called
  (strongly universally) \textit{consistent} if, for all possible distributions
  of $X$ and $Y$, the $L_{2}$-error converges almost surely to 0, that is
$$
\mathbb{E}[m_{n}(X) - m(X)]^2 \underset{a.s.}{\rightarrow} 0
$$
where the expectation is with respect to the query point $X$ and the
samples $D_n$.
\end{mydef}


\section{Consistency of Random Forests}

We consider two different settings. In the setting of Theorem 1, cells may
contain more than one point. Theorem 2 works in the setting of fully-grown
trees, that is, each cell contains exactly one point. 

\begin{mainhyp}[1]
  Throughout, we will assume that the response follows an \textit{additive
    model}, i.e.
  $$
  Y=\sum_{j=1}^p m_j\left(\mathbf{X}^{(j)}\right)+\varepsilon
  $$
  where $\mathbf{X}=\left(\mathbf{X}^{(1)}, \ldots, \mathbf{X}^{(p)}\right)$ is
  uniformly distributed over $[0,1]^p$ and each $m_{j}$ is continuous.
  $\varepsilon$ is an independent, centered Gaussian noise. 
\end{mainhyp}

Hypothesis 1 (H1) will be used mostly to relate the estimate to the true response via
a noise term and consequently exploit the fact that the noise $\varepsilon$ is
assumed to satisfy $\mathbb{E}\left[ \varepsilon \right] = 0$ and
$\text{Var}(\varepsilon) = \sigma^2$. We could, in fact, also simply work with
bounded noise with zero mean.


An essential tool for both theorems is Proposition 2, which states that the
variation of $m$ within a cell goes to $0$ as $n$ grows. This will be used to
address the estimation error.

\begin{mydef}[Variation]
  The \textit{variation of $m$ within cell $A$} is given as
  $$\Delta(m, A):=\sup _{\mathbf{x}, \mathbf{x}^{\prime} \in A}\left|m(\mathbf{x})-m\left(\mathbf{x}^{\prime}\right)\right|$$
\end{mydef}

\begin{myprop}[2 (\cite{scornet_ConsistencyRandomForests_2015})]
  Let $A_n(X, \Theta)$ be the cell that contains $X$ in a tree generated via $\Theta$.
  Assume that (H1) holds. Then, for all $\rho, \xi>0$, there exists $N \in
  \mathbb{N}$ such that, for all $n>N$,
$$
\mathbb{P}\left[\Delta\left(m, A_n(X, \Theta)\right) \leq \xi\right]
\geq 1-\rho
$$
\end{myprop}

Intuitively, this means that the variation of $m$ is small, provided that $n$ is
large enough.



\subsection{Fully-grown trees (Theorem 2)}


Since, by
definition, the tree
estimate for a query point $x$ is the mean response of the cell of $x$,
We
can express the forest estimate aswell by a local averaging formulation.
Starting from the finite forest estimate, we have (see also \cite{biau_RandomForestGuided_2016})
\begin{align*}
  m_{M, n}(x, \Theta_{1..M}) 
  &= \frac{1}{M} \sum_{j=1}^M \sum_{i=1}^n \frac{\mathbb{1}_{x_{i} \in A_{n}(x, \Theta_{j})}}{\lvert \lvert A_{n}(x, \Theta_{j}) \rvert \rvert } Y_i \\
  &= \sum_{i=1}^n W_{M,ni}(x) Y_{i}
    \hspace{1cm}
    \text{
    for $W_{M,ni}(x) := \frac{1}{M} \sum_{j=1}^M \frac{\mathbb{1}_{x_{i} \in A_{n}(x, \Theta_{j})}}{\lvert \lvert A_{n}(x, \Theta_{j}) \rvert \rvert}$
    }
\end{align*}

In Theorem 2, we assume that each cell contains exactly one point. Thus,
motivated by the above discussion, the infinite forest estimate can be written as
$$
m_{n}(X) = \sum_{i=1}^n W_{ni}(X)Y_{i} \hspace{1cm} \text{for} \hspace{0.25cm} W_{ni}(X) := \mathbb{E}_{\Theta}\left[ \mathbb{1}_{X_{i}\in A_{n}(X, \Theta)}\right]
$$

In both cases, the weights sum to one. Note that $W_{ni}$ is the expectation over
tree randomness $\Theta$ and nothing else but the
probability that $x_{i}$ and $x$ are connected in a tree generated with
$\Theta$. The error of the estimate can be bounded from above by
\begin{align}
\mathbb{E}\left[m_{n}(X) - m(X) \right]^2 \leq 
2 \mathbb{E}\left[ \sum_{i=1}^n W_{ni}(X)(Y_{i}-m(X_{i})) \right]^2 + 
2 \mathbb{E}\left[ \sum_{i=1}^n W_{ni}(X)(m(X_{i})-m(X)) \right]^2 
  \label{eq:decomp}
\end{align}
The first term on the right-hand-side can be considered a kind of
approximation error and the second term a kind of estimation error.
(since $\sum_{i=1}^n W_{ni}(X)m(X)$ is the best possible estimate)
To show consistency, we will show the convergence of each of the two terms separately.

\paragraph{Approximation error} Let us first turn towards the second term. Note
that it depends on the difference between the responses for two points. Indeed,
we only need to consider situations in which the two points are in the same
cell, enabling us to apply Proposition 2, which bounds the variation of $m$ within
a cell. Using Jensen's inequality and the definition of $\Delta$, it holds that
$$
\mathbb{E}\left[ \sum_{i=1}^n W_{ni}(X)(m(X_{i})-m(X)) \right]^2 \leq
\mathbb{E}\left[ \mathbb{1}_{x_{i} \in A_{n}(x, \Theta)} (m(X_{i}) - m(X))^2 \right]
\leq
\mathbb{E}\left[\Delta^2(m, A_{n}(x, \Theta)\right]
$$
which, according to Proposition 2 is bounded from above by $\alpha(4 \lvert \lvert
m \rvert \rvert_{\infty} + 1)$ for sufficiently large $n$, where $\alpha > 0$ is arbitrary.

\paragraph{Estimation error} Let us now rest our attention on the first term in
(\ref{eq:decomp}). We denote the noise as $\varepsilon_{i} = Y_{i}-m(X_{i})$ and the
indicator that points are connected to the query point in some tree as $Z_{i} :=
\mathbb{1}_{X_{i} \in A_{n}(X, \Theta)}$ (resp. $Z'_{j} := \mathbb{1}_{X_{i}\in
  A_{n}(X, \Theta')}$). It holds that
$$
\mathbb{E}\left[ \sum_{i=1}^n W_{ni}(X)\varepsilon_{i} \right]^2 
= \mathbb{E}\left[
\sum_{i,j=1}^n   W_{ni}(X)  W_{nj}(X) \varepsilon_{i} \varepsilon_{j}
\right]
= 
 \mathbb{E}\left[ \sum_{i=1}^n  W_{ni}(X)^2 \varepsilon_{i}^2  \right] 
 +  \mathbb{E}\left[ \sum_{i \not= j} Z_{i} Z'_{j} \varepsilon_{i} \varepsilon_{j}  \right]
$$
Note that the left term considers single trees, whereas
the right term considers interactions between two trees generated with $\Theta$
and $\Theta'$, respectively.

\paragraph{Single Trees} To find a vanishing upper bound for the left term,
notice that we have assumed the noise $\varepsilon_{i}$ to be bounded in (H1).
We can bound the probability of connection $W_{ni}(X)$ with a simple counting
argument. Informally, since
\begin{align*}
W_{ni}(X) &=  \mathbb{E}\left[ Z_{i} \right]  
  % &= \mathbb{P}( X, X_i \text{selected in subsampling step} \land X, X_i \text{connected in tree generated with} \Theta)
= \mathbb{P}(\text{$X$, $X_{i}$ selected in subsampling step $\land$ $X, X_{i}$ connected in tree generated via $\Theta$}) \\
&\leq
\mathbb{P}(\text{$X$, $X_{i}$ selected in subsampling step}) \\
&= \frac{{a_{n}-1\choose n_{1}}}{{a_{n} \choose n}} = \frac{a_{n}}{n}
\end{align*}

Putting this together, assuming that  $\frac{a_n}{n} \to 0$, we have
$$
\mathbb{E}\left[ \sum_{i=1}^n W_{ni}(X) \varepsilon_{i}^2 \right] \leq
\mathbb{E}\left[ \max_{l \in [n]} W_{nl}(X) \max_{i \in [n]} \varepsilon_{i}^2
\right] \leq \frac{a_{n}}{n}C \to 0
 $$

 \paragraph{Tree interactions} There are two different assumptions (H2.1) and
 (H2.2) that we can make that will allow us to bound the tree interaction term.
 Here, we wil only treat the proof based on (H2.2). By the rule of iterated
 expectation, we have

 \begin{align}
 \mathbb{E}\left[ \sum_{i \not= j} Z_{i} Z'_{j} \varepsilon_{i} \varepsilon_{j} \right]  = 
  \mathbb{E}\left[ \sum_{i \not= j} Z_{i} Z'_{j} \varepsilon_{i} 
   \mathbb{E}\left[ \varepsilon_{j} ~|~ X_{i}, X_{j}, Z_{i}, Z'_{j}, Y_{i} \right] 
   \right]
   \label{eq:tree-interactions}
 \end{align}
We will handle the inner and outer expectations separately. To assert their
boundedness, we will have to impose two assumptions (H2.2a and H2.2b).

\paragraph{Inner expectation}  In general, for random variables $A, B, C$ (where $B$ is discrete), it holds that
$$
 \mathbb{E}\left[ A ~|~B, C \right]  = \sum_{b \in \text{range}(B)}  \frac{\mathbb{E}\left[ A \mathbb{1}_{B=b} ~|~ C \right]}{\mathbb{P}(B=b ~|~C)} \mathbb{1}_{B=b}
$$
Applied to eq. (\ref{eq:tree-interactions}), this yields
\begin{align}
  \mathbb{E}\left[ \varepsilon_{j} ~|~ X_{i}, X_{j}, Z_{i}, Z'_{j}, Y_{i} \right] 
  = 
\sum_{k \in \{ 0,1 \}^2}  \frac{\mathbb{E}\left[ \varepsilon_{i}
    \mathbb{1}_{(Z_{i}, Z'_{j})=k} ~|~ X_{i}, X_{j}, Y_{i}
  \right]}{\mathbb{P}((Z_{i}, Z'_{j})=k ~|~ X_{i}, X_{j}, Y_{i})}
\mathbb{1}_{(Z_{i}, Z'_{j})=k}
\label{eq:b}
\end{align}
This is useful since, in fact, $\mathbb{E}\left[ \varepsilon_{i}
  \mathbb{1}_{(Z_{i}, Z'_{j})=k} ~|~ X_{i}, X_{j}, Y_{i} \right] =
\text{Cov}(\varepsilon_{1}, \mathbb{1}_{(Z_{i}, Z'_{j})=k} ~|~ X_{i}, X_{j},
Y_{j})$ due to that $\mathbb{E}\left[ \varepsilon_{i} ~|~ X_{i}, X_{j}, Y_{j}
\right] = 0$ (by H1). Note that, if $\varepsilon_{i}$ and $(Z_{i}, Z'_j)$ are
independent, this term and any covariance or correlation based on it will be
zero. However, in Random Forests, the partitions -- and thus $(Z_{i}, Z'_{j})$ -- depend
on the responses $Y_{i}$ due to optimising for purity in the CART Criterion
during tree construction. Consequently, we will formulate an assumption that
will express the boundedness of this correlation, and consequently the
expectation. Based on the definition of correlation, we obtain
$$
\text{Corr}( \varepsilon_{i}, \mathbb{1}_{(Z_i, Z'_j)=k} ~|~ X_{i}, X_{j}, Y_{j}  ) 
= \frac{\text{Cov}(\varepsilon_{i}, \mathbb{1}_{(Z_i, Z'_j)=k}  ~|~ X_{i}, X_{j}, Y_{j} )}{
\text{Var}^{1/2}(\varepsilon_{i} ~|~ X_{i}, X_{j}, Y_{j}) \text{Var}^{1/2}(\mathbb{1}_{(Z_i, Z'_j)=k}  ~|~ X_{i}, X_{j}, Y_{j})
}
$$
where $\text{Var}^{\nicefrac{1}{2}}(\varepsilon_{i} ~|~ X_{i}, X_{j}, Y_{j}) =
\sigma$ and $\text{Var}^{1/2}(\mathbb{1}_{(Z_i, Z'_j)=k} ~|~ X_{i}, X_{j},
Y_{j}) \leq \mathbb{P}((Z_{i}, Z_{j})=k)$.

Consequently,
$$
 \mathbb{E}\left[ \varepsilon_{i} \mathbb{1}_{(Z_i, Z'_j)=k} ~|~ X_{i}, X_{j}, Y_{j} \right] 
 \leq \text{Corr}( \varepsilon_{i}, \mathbb{1}_{(Z_i, Z'_j)=k ~|~ X_{i}, X_{j}, Y_{j}} )  
 \mathbb{P}^{1/2}((Z_{i}, Z_{j})=k)
 \sigma
$$

which, applied to (\ref{eq:b}), yields

$$
\mathbb{E}\left[ \varepsilon_{j} ~|~ X_{i}, X_{j}, Z_{i}, Z'_{j}, Y_{i} \right] 
 \leq \sigma \cdot 4 \max_{k \in \{ 0,1 \}^2} \frac{|\text{Corr}(\varepsilon_{i}, \mathbb{1}_{(Z_i, Z'_j)=k} ~|~ X_{i}, X_{j}, Y_{j})|}{
\mathbb{P}^{1/2}({(Z_i, Z'_j)=k} )
}
$$
This motivates assumption (H2.2a), stated as
$$
\max_{k \in \{ 0,1 \}^2} \frac{|\text{Corr}(\varepsilon_{i}, \mathbb{1}_{(Z_i, Z'_j)=k} ~|~ X_{i}, X_{j}, Y_{j})|}{
\mathbb{P}^{1/2}({(Z_i, Z'_j)=k} )}
\leq \gamma_{n}
$$
for sufficiently large $n$ and a sequence $\{\gamma_{n}\}_n \to 0$. In summary,
this means that (H2.2a) implies that, almost surely
$$
\mathbb{E}\left[ \varepsilon_i ~|~ Z_i, Z'_j, X_i, X_j, Y_j \right] \leq 4
\sigma \gamma_n
$$

% eODO (additional) intuition on this

\paragraph{Outer Expectation} If (H2.2a) is given, we have, by linearity of expectation
\begin{align}
 \mathbb{E}\left[ \sum_{i \not= j} Z_{i} Z'_{j} \varepsilon_{i} \varepsilon_{j} \right]  = 
  \mathbb{E}\left[ \sum_{i \not= j} Z_{i} Z'_{j} \varepsilon_{i} 
   \mathbb{E}\left[ \varepsilon_{j} ~|~ X_{i}, X_{j}, Z_{i}, Z'_{j}, Y_{i} \right] 
   \right] 
   \leq \gamma_{n} \sum_{i=1}^n  \mathbb{E}\left[ Z_{i} \varepsilon_{i} \right] 
   \label{eq:outer-exp}
\end{align}
This is where we extract another assumption. We
use the law of iterated expectation to factor out an expection of the noise
term:
$$
\dots \leq \gamma_{n} \sum_{i=1}^n  \mathbb{E}\left[ Z_i  \mathbb{E}^{\nicefrac{1}{2}}\left[ |\varepsilon_{i}|^2 ~|~ X_{i}, Z_{i} \right]  \right] 
$$
Similar to previously, one can then derive assumption (H2.2b) and show that for
some constant $C > 0$
$$
\max_{k \in \{ 0,1 \}} \frac{|\text{Corr}(\varepsilon_{i}^2, \mathbb{1}_{Z_{i}=k} ~|~ X_{i})|}{
\mathbb{P}^{1/2}(Z_{i}=k ~|~ X_{i})
}
\leq C
\hspace{0.5em} \Rightarrow \hspace{0.5cm}
 \mathbb{E}\left[ |\varepsilon_{i}|^2 ~|~ X_{i}, Z_{i} \right] \leq 4C \sigma^2
$$
Assuming this holds, we obtain, continuing from eq. (\ref{eq:outer-exp})
$$
\gamma_{n} \sum_{i=1}^n \mathbb{E}\left[ Z_i \mathbb{E}^{\nicefrac{1}{2}}\left[
    |\varepsilon_{i}|^2 ~|~ X_{i}, Z_{i} \right] \right] \leq \gamma_{n} 2
C^{\nicefrac{1}{2}} \sigma \sum_{i=1}^n \mathbb{E}\left[ Z_{i} \right] \leq \gamma_{n} 2
C^{\nicefrac{1}{2}} \sigma
$$
Which concludes the proof of (H2.2) $\Rightarrow$ Theorem 2. In summary, the
full statement is given below.


\begin{mainhyp}[2.2]
  There exist a constant $C>0$ and a sequence $\left(\gamma_n\right)_n
  \rightarrow 0$ such that, almost surely, the following two statements hold:
  \begin{align*}
                \max _{\ell_1, \ell_2=0,1} \frac{\left|\operatorname{Corr}\left(Y_i-m\left(\mathbf{X}_i\right), \mathbb{1}_{Z_{i, j}=\left(\ell_1, \ell_2\right)} \mid \mathbf{X}_i, \mathbf{X}_j, Y_j\right)\right|}{\mathbb{P}^{1 / 2}\left[Z_{i, j}=\left(\ell_1, \ell_2\right) \mid \mathbf{X}_i, \mathbf{X}_j, Y_j\right]} \leq \gamma_n \tag{H2.2a}\\
              \max _{\ell_1=0,1} \frac{\left|\operatorname{Corr}\left(\left(Y_i-m\left(\mathbf{X}_i\right)\right)^2, \mathbb{1}_{Z_i=\ell_1} \mid \mathbf{X}_i\right)\right|}{\mathbb{P}^{1 / 2}\left[Z_i=\ell_1 \mid \mathbf{X}_i\right]} \leq C . \tag{H2.2b}
  \end{align*}
 \end{mainhyp}


\begin{mainthm}[2 (\cite{scornet_ConsistencyRandomForests_2015})]
  Assume that (H1) and (H2.2) are satisfied, and let $t_n=a_n$. Then, provided
  $a_n \rightarrow \infty, t_n \rightarrow \infty$ and $a_n \log n / n
  \rightarrow 0$, random forests are consistent, that is,
  $$
  \lim _{n \rightarrow \infty}
  \mathbb{E}\left[m_n(\mathbf{X})-m(\mathbf{X})\right]^2=0
  $$
\end{mainthm}

Let us briefly discuss an alternate set of
assumptions. Again, it comes into action when handling the tree interaction term
$\mathbb{E}\left[ \sum_{i\not= j} Z_{i} Z'_{j} \varepsilon_{i} \varepsilon_{j}
\right]$. Recall that $Z_{i} = 1$ if and only if $X_{i}$ is connected to the query point $X$ in
a tree generated via $\Theta$. We bring in the additional observation that,
since each cell contains exactly one point, $x_{i}$ can only be connected to the
query point $x$ if there are no other points in the hyperrectangle defined by
$x_i$ and $x$. In this case, we call $x_{i}$ and $x$ \textit{layered nearest
  neighbours} (LNNs). Previous work gives us a bound on the expected number of
LNNs of a random query point, denoted by $l_{a_{n}}(X)$ (see
\cite{scornet_ConsistencyRandomForests_2015}). Let $L_{i}$ be the indicator of
the event that $X_{i}$ is an LNN to $X$ in a tree generated via $\Theta$ and
likewise for $L_{j}'$. Let ${\psi_{i, j}=\mathbb{E}\left[Z_i Z_j^{\prime} \mid
    \mathbf{X}, \Theta, \Theta^{\prime}, \mathbf{X}_1, \ldots,
    \mathbf{X}_n\right]}$ and ${\psi_{i, j}\left(Y_i,
    Y_j\right)=\mathbb{E}\left[Z_i Z_j^{\prime} \mid \mathbf{X}, \Theta,
    \Theta^{\prime}, \mathbf{X}_1, \ldots, \mathbf{X}_n, Y_i, Y_j\right]}$. One
can show that
\begin{align*}
\mathbb{E}\left[ \sum_{i \not= j} Z_{i}, Z'_{j} \varepsilon_{i} \varepsilon_{j}
\right] & \leq \mathbb{E}\left[ \sum_{i \not= j} ~|\varepsilon_{i}|~
  |\varepsilon_{j}|~ L_{i} L'_{j} ~|\psi_{ij}(Y_{i}, Y_{j}) - \psi_{ij}|~ \right] \\
& \leq
\mathbb{E}\left[ \max_{i \in [n]} |\varepsilon_{i}|^2 ~ \max_{i \not= j}
  |\psi_{ij}(Y_{i}, Y_{j}) - \psi_{ij}| ~\sum_{i \not= j}  L_{i} L'_{j}\right]
\end{align*}
The first factor vanishes since by (H1) noise is assumed to be bounded. The
third factor is bounded by the total number of LNNs $l_{a_{n}}$, seeing that
$\sum_{i \not= j} L_{i}, L'_{j} \leq l_{a_{n}}^2(X)$. The second factor is assumed
to vanish by means of (H2.1).

\begin{mainhyp}[2.1] If $\varepsilon$ is a bounded random variable,
  (H2.1) can be stated as
  $$
  \lim _{n \rightarrow \infty} \mathbb{E}\left[\max _{\substack{i, j \\ i \neq
        j}}\left|\psi_{i, j}\left(Y_i, Y_j\right)-\psi_{i, j}\right|\right]^2=0
  $$
  If we assume Gaussian noise, the above term requires an extra factor of $(log a_n)^{2 p-2}(\log n)^2$.
\end{mainhyp}



\subsection{Trees not fully grown (Theorem 1)}



Theorem 1 acts in the setting of leaves containing more than one point.

\begin{mainthm}[1 (\cite{scornet_ConsistencyRandomForests_2015})]
  Assume that $(\mathrm{H} 1)$ is satisfied. Then, provided $a_n \rightarrow
  \infty, t_n \rightarrow \infty$ and $t_n\left(\log a_n\right)^9 / a_n
  \rightarrow 0$, random forests are consistent, that is,
  $$
  \lim _{n \rightarrow \infty} \mathbb{E}\left[m_n(\mathbf{X})-m(\mathbf{X})\right]^2=0
  $$
\end{mainthm}

Figure \ref{fig:thm-1-overview} illustrates the individual statements that yield
Theorem 1. The \textit{theoretical Random Forest} is one built by optimizing a
CART criterion that is not based on empirical variance but instead on the general
definition of variance in the limit.
Trees in a theoretical Random Forest do not depend on a given dataset $D_n$, but
are still random due to the random selection of candidate split dimensions. A theoretical
random forest is assumed to be of fixed depth.


\begin{figure*}[h]
  \centering
  \includegraphics[width=\textwidth]{thm-1-overview.pdf}
  \caption{Overview of the proof of Theorem 1 (omitting Technical Lemma 1 and Lemma 2). }
  \label{fig:thm-1-overview}
\end{figure*}


We refrain from illustrating the entire proof of Theorem 1 but nevertheless
attempt to shed light on the overall strategy and some key ideas. 
% The proof of Theorem 1 rests
% on theory developed in \cite{gyorfi}.
Recall that in finding a regression
function estimate, we restrict ourselves to a hypothesis class $\mathcal{F}_n$.
The following Lemma gives a bound on the $L_2$-risk of the estimate in terms of
estimation and approximation error with respect to $\mathcal{F}_n$.

\begin{lemma}[10.1 (\cite{gyorfi_DistributionFreeTheoryNonparametric_2002})]
  Let
  $\mathcal{F}_n$ be a class of functions $f:
  \mathbb{R}^p \to \mathbb{R}$ depending on the data
  $D_n$. 
  Then,
  \begin{align*}
    \mathbb{E}\left[ \left|m_n(x)-m(x)\right|^2 \right] \leq
    & ~ 2 \sup _{f \in \mathcal{F}_n}
      \left|
      \frac{1}{n}
      \sum_{j=1}^n
      \left| f\left(X_j\right)-Y_j\right|^2-\mathbb{E}\left[(f(X)-Y)\right]^2
      \right| \\
      & +\inf _{f \in \mathcal{F}_n} \mathbb{E}\left[\left|f(x)-m(x)\right|^2  \right]
  \end{align*}
\end{lemma}



\paragraph{Estimation error} The first term quantifies the difference between
the empirical $L_2$-risk and the true $L_2$-risk. This is reminiscent of the
notion of estimation error. The bound for this term rests on theory developed in
\cite{gyorfi_DistributionFreeTheoryNonparametric_2002}. To enable this, we have
to take a detour via first considering the truncated estimate. Luckily, it turns
out that we can conclude the consistency of the untruncated estimate from there.
To bound the estimation error, we relate it to characteristics of the induced
partition (\cite{gyorfi_DistributionFreeTheoryNonparametric_2002} Thm 9.1).
These characteristics
are indirectly controlled by the rates required by the theorem.
% TODO (additional) how so?


\paragraph{Approximation error} The second term is the classical approximation
error. To handle this term, the basic idea is that any Random Forest regression
function will be cell-wise constant. So, the approximation error will compare
the difference of the cell-wise estimate with the true estimate at a given
point. This is bounded by the variation of $m$, enabling the application of
Proposition 2.


% Given the right conditions on $\mathcal{F}_n$, one
% can show that the empirical $L_2$-risk is in fact (uniformly) close to the true
% $L_2$ risk.



%
% The second term is the classical approximation error.
%
% To show consistency, it suffices to show that both terms converge to zero.
% However, in particular for handling the estimation error, we have to take a
% detour by first considering the truncated estimate. Luckily, it turns out that
% we can conclude the consistency of the untruncated estimate from there.

% For a positive sequence
% $(\beta_{n})_{n}$, we define the **truncated-operator** $T_{\beta_{n}}$ by
% $$T_{\beta_n} u=\begin{cases}u, & \text { if }|u|<\beta_n, \\ \operatorname{sign}(u) \beta_n, & \text { if }|u| \geq \beta_n .\end{cases}$$
% Of particular interest are the truncation of regression fn to $\beta_{n}$:   $T_{\beta_n} m_n(\mathbf{X}, \Theta)$
% ; the truncation of outcomes to $L$: $Y_L=T_L Y$; and the truncation of $i$-th
% output to $L$:  $Y_{i, L}=T_L Y_i$.

% \begin{mainthm}[Gyorfi, Thm 10.2]
%   Let $\mathcal{F}_n(\Theta)$ be the set of all functions $f:[0,1]^d \rightarrow
%   \mathbb{R}$ piecewise constant on each cell of the partition
%   $\mathcal{P}_n(\Theta)$. If the following holds:
% \begin{enumerate}[(i)]
% \item $\lim _{n \rightarrow \infty} \beta_n=\infty$
%   \item $\lim _{n
%     \rightarrow \infty} \mathbb{E}\left[\inf _{f \in
%       \mathcal{F}_n(\Theta),\|f\|_{\infty} \leq \beta_n}
%     \mathbb{E}_{\mathbf{X}}[f(\mathbf{X})-m(\mathbf{X})]^2\right]=0$
%   \item for all $L>0$, 
%   $$
%   \lim _{n \rightarrow \infty} \mathbb{E}\left[\sup _{\substack{f \in \mathcal{F}_n(\Theta) \\\|f\|_{\infty} \leq \beta_n}}\left|\frac{1}{a_n} \sum_{i \in \mathcal{I}_{n, \Theta}}\left[f\left(\mathbf{X}_i\right)-Y_{i, L}\right]^2-\mathbb{E}\left[f(\mathbf{X})-Y_L\right]^2\right|\right]=0 .
%   $$
% \end{enumerate}
% Then
% $$
% \lim _{n \rightarrow \infty} \mathbb{E}\left[T_{\beta_n} m_n(\mathbf{X}, \Theta)-m(\mathbf{X})\right]^2=0
% $$
% \end{mainthm}

% Here, (ii) corresponds to the approximation error and (iii) to the estimation
% error of the truncated estimate. To handle the approximation error, the basic
% idea is that any regression function will be cell-wise constant. So, the
% approximation (??) error will compare the difference of the cell-wise estimate with the
% true estimate at a given point. This is bounded by the variation of $m$,
% enabling the application of Proposition 2. 

% \section{Discussion}
% remains to be seen how tree interaction terms, covariance is related to
% diversity, entropy etc



% biblatex, see usepackages above
\printbibliography  

\end{document}
